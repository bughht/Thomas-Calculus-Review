\documentclass{article}
\usepackage{amsmath}
\usepackage{bm}
\usepackage{geometry}
\usepackage{pgfplots}
\geometry{a4paper,left=3.5cm,right=3.5cm,top=3cm,bottom=4cm}
\title{Thomas Calculus Review}
\author{Harryhht}
\date{\today}

\begin{document}
    \maketitle
    \newpage
    \tableofcontents
    \newpage
    \section{Function}
        \subsection{Function and their Graphs}
            \textit{DEFINITION:
            A function $f$ from a set $D$ to a set $Y$ is a rule that assigns a unique (single) element $f(x)\in Y$ to each element $x \in D$.}
            \paragraph{Linear Functions} $f(x)=mx+ba$
            \paragraph{Power Functions} $f(x)=x^a$ where $a$ is a constant
            \paragraph{Polynomials} $p(x)=a_n x^n+ a_{n-1} x^{n-1}+a_{n-2} x^{n-2}+\cdots+a_1 x + a_0$ where n is a nonnegative integer and $a_0, a_1,\cdots a_n$ are real constants (called the \textbf{coefficients} of the polynomial) 
            \paragraph{Rational Functions} $f(x)=\cfrac{p(x)}{q(x)}$ where $p$ and $q$ are polynomials.
            \paragraph{Algebraic Functions} Any function constructed from polynomials using algebraic operations lies within the class of \textbf{algebraic functions}.
            \paragraph{Trigonometric Function}
            \begin{equation}
                \left\{
                \begin{aligned}
                    f(x)=\sin(x)\\
                    f(x)=\cos(x)\\
                    f(x)=\tan(x)\\
                    f(x)=\csc(x)\\
                    f(x)=\sec(x)\\
                    f(x)=\cot(x)\\
                \end{aligned}
                \right.
            \end{equation}
            \paragraph{Exponential Functions} $f(x)=a^x$ where the base $a$ is a positive constant and $a\ne 1$
            \paragraph{Logarithmic Functions} $f(x)=\log _{a}x$ 
            \paragraph{Transcendental Functions} Functions that are not algebraic.

        \subsection{Combining Functions; Shifting and Scaling Graphs}
            \paragraph{Sums,Differences,Products, and Quotients}
            \begin{equation}
                \begin{cases}
                    (f+g)(x)=f(x)+ g(x)\\
                    (f-g)(x)=f(x)- g(x)\\
                    (fg)(x)=f(x)g(x)\\
                    (\cfrac{f}{g})(x)=\cfrac{f(x)}{g(x)}\\
                    (cf)(x)=cf(x)
                \end{cases}
            \end{equation} 

            \paragraph{Composite Functions}
            \subparagraph{DEFINITION} If $f$ and $g$ are functions, the \textbf{composite} of function $f\circ g$ ("$f$ composed with $g$") is defined by
            \begin{equation}
                (f\circ g)(x)=f(g) 
            \end{equation}

            \paragraph{Shifting a Graph of a Function}
            Vertical Shifts and Horizontal Shifts
            
            \paragraph{Scaling and Reflecting a Graph of a Function}
            Vertical and Horizontal Scaling and Reflecting Formulas
        \subsection{Trigonometric Function}

            \paragraph{Angles} are measured in degrees or radians.\\
            1 radian $= \cfrac{180}{\pi} (\approx 57.3)$degrees
            
            \paragraph{The Six Basic Trigonometric Functions}
            \begin{equation}
                \begin{cases}
                    \sin \theta=\cfrac{y}{r}\\
                    \csc \theta=\cfrac{r}{y}\\
                    \cos \theta=\cfrac{x}{r}\\
                    \sec \theta=\cfrac{r}{x}\\
                    \tan \theta=\cfrac{y}{x}\\
                    \cot \theta=\cfrac{x}{y}
                \end{cases}    
                \begin{cases}
                    \tan \theta=\cfrac{\sin \theta}{\cos \theta}\\
                    \cot \theta=\cfrac{1}{\tan \theta}\\
                    \sec \theta=\cfrac{1}{\cos \theta}\\
                    \csc \theta=\cfrac{1}{\sin \theta}
                \end{cases}
            \end{equation}
            \paragraph{Trigonometric Identities}
            \[x=r\cos \theta \qquad y=r\sin \theta\]
            \[\sin ^2 \theta + \cos ^2 \theta =1\]
            \[1+ \tan ^2 \theta = \sec ^2 \theta\]
            \[1+ \cot ^2 \theta = \csc ^2 \theta\]
            \subparagraph{Addition Formulas}
            \begin{equation}
                \begin{aligned}
                    \cos (A+B)=\cos A \cos B- \sin A \sin B\\
                    \sin (A+B)=\sin A \cos B+ \cos A \sin B
                \end{aligned}
            \end{equation}
            
            \subparagraph{Double-Angle Formulas}
            \begin{equation}
                \begin{aligned}
                    \cos 2\theta &= \cos ^2 \theta - \sin ^2 \theta\\
                    \sin 2\theta &= 2\sin \theta \cos \theta
                \end{aligned}
            \end{equation}

            \subparagraph{Half-Angle Formulas}
            \begin{equation}
                \begin{aligned}
                    \cos ^2 \theta = \cfrac{1+\cos 2\theta}{2}\\
                    \sin ^2 \theta = \cfrac{1-\cos 2\theta}{2} 
                \end{aligned}  
            \end{equation}

            \subparagraph{The Law of Cosines}
            \[c^2=a^2+b^2-2ab\cos \theta\]

            \subparagraph{Two Special Inequalities}
            \[-\lvert \theta \rvert \le \sin \theta \le \lvert \theta \rvert\]
            \[-\lvert \theta \rvert \le 1-\cos \theta \le \lvert \theta \rvert\]

            \paragraph{Transformations of Trigonometric Graphs}
            \[y=af(b(x+c))+d\]
            $a$:Vertical stretch or compression;
            reflection about $y=d$ if negative\\
            $b$:Horizontal stretch or compression;
            reflection about $x=-c$ if negative\\
            $c$:Horizontal shift\\
            $d$:Vertical shift\\

    \newpage
    \section{Limits and Continuity}
        \subsection{Rates of Change and Tangents to Curves} 
            \[\cfrac{\Delta y}{\Delta x}\]
            \paragraph{DEFINITION} The \textbf{average rate of change} of $y=f(x)$ with respect to $x$ over the interval $[x_1,x_2]$ is
            \[\cfrac{\Delta y}{\Delta x}=\cfrac{f(x_2)-f(x_1)}{x_2-x_1}=\cfrac{f(x_1 +h)-f(x_1)}{h},\quad h\ne 0\]
            \subsection{Limit of a Function and Limit Laws}
                \paragraph{Limits of Function Value} $\lim\limits_{x\to c}f(x)=L$ (read ``the limit of $f(x)$ as $x$ approaches $c$ is $L$'')
                    \subparagraph{``Informal'' definition} The values of $f(x)$ are close to the number $L$ whenever is close to $c$ (on either side of $c$)
                \paragraph{The Limit Laws}
                    \subparagraph{Limit Laws} If $L$, $M$,$c$ and $k$ are real numbers and $\lim\limits_{x\to c}f(x)=L$\quad and \quad $\lim\limits_{x\to c} g(x)=M$, then
                    \begin{equation}
                        \begin{aligned}
                            \lim\limits_{x\to c} (f(x)+g(x))&=L+M\\
                            \lim\limits_{x\to c} (f(x)-g(x))&=L-M\\
                            \lim\limits_{x\to c} (f(x)\cdot g(x))&=L\cdot M\\
                            \lim\limits_{x\to c} (k\cdot f(x))&=k\cdot L\\
                            \lim\limits_{x\to c} \cfrac{f(x)}{g(x)}&= \cfrac{L}{M},\quad M\ne 0\\
                            \lim\limits_{x\to c}[f(x)]^n &=L^n\\
                            \lim\limits_{x\to c}\sqrt[n]{f(x)}&=\sqrt[n]{L}=L^{\frac{1}{n}}
                        \end{aligned}
                    \end{equation}
                    \subparagraph{Limits of Polynomials} If $P(x)=a_n x^n+a_{n-1}x^{n-1}+\cdots +a_0$, then 
                    \[\lim\limits_{x\to c}P(x)=P(c)=a_nc^n+a_{n-1}c^{n-1}+\cdots +a_0\]
                    \subparagraph{Limits of Rational Funcitions} If $P(x)$ and $Q(x)$ are polynomials and $Q(c)\ne 0$, then
                    \[\lim\limits_{x\to c} \cfrac{P(x)}{Q(x)}=\cfrac{P(c)}{Q(c)}\]
                \paragraph{The Sandwich Theorem}
                    Suppose that $g(x)\le f(x) \le h(x)$ for all $x$ in some open interval containing $c$, expect possibly at $x=c$ itself. Suppose also that 
                    \[\lim\limits_{x\to c}g(x)=\lim\limits_{x\to c}h(x)=L\]
                    Then $\lim\limits_{x\to c}f(x)=L$. 
                \subparagraph{Theorem} If $f(x)\le g(x)$ for all $x$ in some open interval containing $c$, expect possibly at $x=c$ itself, and the limits of $f$ and $g$ both exist as $x$ approaches $c$, then
                \[\lim\limits_{x\to c}f(x)\le \lim\limits_{x\to c}g(x)\]
            \subsection{The Precise Definition of a Limit}
                \paragraph{DEFINITION} Let $f(x)$ be defined on an open interval about $c$, except possibly at $c$ itself. We say that the \textbf{limit of $f(x)$ as $x$ approaches $c$ is the number $L$}, and write
                \[\lim\limits_{x\to c}f(x)=L\]
                if, for every number $\epsilon>0$, there exists a corresponding number $\delta>0$ such that for all $x$,
                \[0<\lvert x-c \rvert <\delta \quad \Rightarrow \quad \lvert f(x)-L \rvert <\epsilon\]
            \subsection{One-Sided Limits}
                \paragraph{Two-sided limits}right-hand limit and left-hand limit
                    \subparagraph{THEOREM} A function $f(x)$ has a limit as $x$ approaches $c$ if and only if it has left-hand and right-hand limits there and these one-sided limits are equal:
                    \[\lim\limits_{x\to c}f(x)=L\quad \Leftrightarrow \quad \lim\limits_{x\to c^-}f(x)=L\quad \text{and} \quad \lim\limits_{x\to c^+}f(x)=L\]
                    \subparagraph{Limits Involving $(\sin \theta)/\theta$}
                    \[\lim\limits_{\theta\to 0}\cfrac{\sin\theta}{\theta}=1\]
            \subsection{Continuity}
                \paragraph{DEFINITION} Let $c$ be a real number on the $x$-axis.\\
                The function $f$ is \textbf{Continuous at} $c$ if 
                \[\lim\limits_{x\to c}f(x)=f(c)\]
                The function $f$ is \textbf{right-continuous at $c$ (or continuous from the right)} if
                \[\lim\limits_{x\to c^+}f(x)=f(c)\]
                The function $f$ is \textbf{left-continuous at $c$ (or continuous from the left)} if
                \[\lim\limits_{x\to c^-}f(x)=f(c)\]
                \paragraph{Properties of Continuous Functions} If the functions $f$ and $g$ are continuous at $x=c$, then the following algebraic combinations are continuous at $x=c$.
                \begin{equation}
                    \begin{aligned}
                        f+g\\
                        f-g\\
                        k\cdot f\\
                        f\cdot g\\
                        f/g\\
                        f^g\\
                        \sqrt[n]{f}\\
                    \end{aligned}
                \end{equation}
                \paragraph{Composite of Continuous Functions} If $f$ is continuous at $c$ and $g$ is continuous at $f(c)$, then the composite $g\circ f$ is continuous at $c$.
                \paragraph{Limits of Continuous Functions} If $g$ is continuous at the point $b$ and $\lim\limits_{x\to c}f(x)=b$, then 
                \[\lim\limits_{x\to c}g(f(x))=g(b)=g(\lim\limits_{x\to c}f(x)).\]
                \paragraph{The Intermediate Value Theorem for Continuous Functions} If $f$ is a continuous function on a closed interval [$a$,$b$], and if $y_0$ is any value between $f(a)$ and $f(b)$, then $y_0=f(c)$ for some $c$ in [$a$,$b$].
            \subsection{Limits Involving Infinity; Asymptotes of Graphs}
                \[\lim\limits_{x\to \pm \infty}k=k\]
                \[\lim\limits_{x\to \pm \infty}\frac{1}{x}=0\]                    
                \subparagraph{EXAMPLE}
                \[\lim\limits_{x\to -\infty}\cfrac{11x+2}{2x^3-1}
                =\lim\limits_{x\to -\infty}\cfrac{(11/x^2)+(2/x^3)}{2-(1/x^3)}=\cfrac{0+0}{2-0}=0\]
                \paragraph{DEFINITION} A line $y=b$ is a \textbf{horizontal asymptote} of the graph of a function $y=f(x)$ if either
                \[\lim\limits_{x\to \infty}f(x)=b\quad\text{or}\quad\lim\limits_{x\to -\infty}f(x)=b\]
                \paragraph{EXAMPLE}
                \[\lim\limits_{x\to \infty}\sin\cfrac{1}{x}=\lim\limits_{t\to 0^+}\sin t=0\]
                \[\lim\limits_{x\to -\infty}x\sin \cfrac{1}{x}=\lim\limits_{t\to 0^+}\cfrac{\sin t}{t}=1\]
                \[\lim\limits_{x\to \infty}x\sin \cfrac{1}{x}=\lim\limits_{t\to 0^-}\cfrac{\sin t}{t}=1\]
                \begin{equation}
                    \begin{aligned}
                        \lim\limits_{x\to -\infty}\cfrac{2x^5-6x^4+1}{3x^2+x-7}
                        &=\lim\limits_{x\to -\infty}\cfrac{2x^3-6x^2+x^{-2}}{3+x^{-1}-7x^{-2}}\\
                        &=\lim\limits_{x\to -\infty}\cfrac{2x^2(x-3)+x^{-2}}{3+x^{-1}-7x^{-2}}\\
                        &=-\infty
                    \end{aligned}
                \end{equation}

    \newpage
    \section{Derivatives}
        \subsection{Tangents and the Derivative at a Point}
            \paragraph{DEFINITIONS} The \textbf{derivative of a function $f$ at a point $x_0$}, denoted $f'(x_0)$, is 
            \[m=\lim\limits_{h\to 0}\cfrac{f(x_0+h)-f(x_0)}{h}\quad\text{(provided the limit exists)}\]
            The \textbf{tangent line} to the curve at $P$ is the line through $P$ with this slope.
        \subsection{The Derivative as a Function}
            \paragraph{Calculating Derivatives from the Definition}
            The process of calculating a derivative is called differentiation. To emphasize the idea that differentiation is an operation performed on a function $y=f(x)$, we use the notation
            \[\frac{d}{dx}f(x)\]
            as another way to denote the derivative $f'(x)$.
            \subparagraph{Notations}
            \[f'{x}=y'=\cfrac{dy}{dx}=\cfrac{df}{dx}=\cfrac{d}{dx}f(x)=D(f)(x)=D_x f(x)\]
            \paragraph{Differentiable on an Interval; One-Sided Derivatives}
            \begin{equation}
                \begin{aligned}
                    \lim\limits_{h\to 0^+}\cfrac{f(a+h)-f(a)}{h}\quad&\textbf{Right-hand derivative at a}\\
                    \lim\limits_{h\to 0^-}\cfrac{f(h+h)-f(b)}{h}\quad&\textbf{Left-hand derivative at b}
                \end{aligned}  
            \end{equation}
            \textbf{When Does a Function \textit{Not} Have a Derivative at a Point?}\\
            1. a corner,where the one-sided derivatives differ.\\
            2. a cusp, where the slope of \textit{PQ} approaches $\infty$ from one side and $-\infty$ from the other.\\
            3. a vertical tangent, where the slope of $PQ$ approaches $\infty$ from both sides or approaches $-\infty$ from both sides.\\
            4. a \textit{discontinuity}.
            \paragraph{Differentiable Functions Are Continuous}
            \subparagraph{Differentiability Implies Continuity} If $f$ has a derivative at $x=c$, then $f$ is continuous at $x=c$.
        \subsection{Differentiation Rules}
            \paragraph{Power,Multiples, Sums, and Differences}
            \begin{equation}
                \begin{aligned}
                    \cfrac{d}{dx}(c)&=0\\
                    \cfrac{d}{dx}(x^n)&=nx^{n-1}\\
                    \cfrac{d}{dx}(cu)&=c\cfrac{du}{dx}\\
                    \text{so }(\cfrac{d}{dx}(cx^n)&=cnx^{n-1})\\
                    \cfrac{d}{dx}(u+v)&=\cfrac{du}{dx}+\cfrac{dv}{dx}\\
                    \cfrac{d}{dx}(uv)&=u\cfrac{dv}{dx}+v\cfrac{du}{dx}\\
                    \cfrac{d}{dx}(\cfrac{u}{v})&=\cfrac{v\cfrac{du}{dx}-u\cfrac{dv}{dx}}{v^2}\\
                \end{aligned}  
            \end{equation}
            \paragraph{Second- and Higher-Order Derivatives}
            \[f^{''}(x)=\cfrac{d^2y}{dx^2}=\cfrac{d}{dx}(\cfrac{dy}{dx})=\cfrac{dy'}{dx}=y^{''}=D^2(f)(x)={D_x}^2f(x)\]
        \subsection{The Derivative as a Rate of Change}
            \paragraph{Motion Along a Line: Displacement, Velocity, Speed, Acceleration, and Jerk}
            \[v(t)=\cfrac{ds}{dt}\]
            \[\text{Speed}=\lvert v(t) \rvert= \lvert \cfrac{ds}{dt} \rvert\]
            \[a(t)=\cfrac{dv}{dt}=\cfrac{d^2 s}{dt^2}\]
            \[j(t)=\cfrac{da}{dt}=\cfrac{d^3 s}{dt^3}\]
        \subsection{Derivatives of Trigonometric Functions}
            \begin{equation}
                \begin{aligned}
                    \cfrac{d}{dx}(\sin x)&=\cos x\\
                    \cfrac{d}{dx}(\cos x)&=-\sin x\\
                    \cfrac{d}{dx}(\tan x)&=\sec ^2 x\\
                    \cfrac{d}{dx}(\cot x)&=-\csc ^2 x\\
                    \cfrac{d}{dx}(\sec x)&=\sec x \tan x\\
                    \cfrac{d}{dx}(\csc x)&=-\csc x \cot x\\
                \end{aligned} 
            \end{equation}
        \subsection{The Chain Rule*}
            \[\cfrac{dy}{dx}=\cfrac{dy}{du}\cdot\cfrac{du}{dx}\]
            \[(f\circ g)'(x)=f'(g(x))\cdot g'(x)\]
            \paragraph{Power Chain Rule} Using chain rules in loop.
        \subsection{Implicit Differentiation}
            \paragraph{Implicitly Defined Functions}
                \subparagraph{EXAMPLE 1}
                    \begin{equation}
                        \begin{aligned}
                            y^2&=x\\
                            2y\cfrac{dy}{dx}&=1\\
                            \cfrac{dy}{dx}&=\cfrac{1}{2y}
                        \end{aligned}  
                    \end{equation}
                \subparagraph{EXAMPLE 2}
                    \begin{equation}
                        \begin{aligned}
                            x^2+y^2&=25\\
                            \cfrac{d}{dx}(x^2)+\cfrac{d}{dx}(y^2)&=\cfrac{d}{dx}(25)\\
                            2x+2y\cfrac{dy}{dx}&=0\\
                            \cfrac{dy}{dx}&=-\cfrac{x}{y}
                        \end{aligned} 
                    \end{equation}

                \subparagraph{EXAMPLE 3}
                    \begin{equation}
                        \begin{aligned}
                            y^2&=x^2+\sin xy\\
                            \cfrac{d}{dx}(y^2)&=\cfrac{d}{dx}(x^2)+\cfrac{d}{dx}(\sin xy)\\
                            2y\cfrac{dy}{dx}&=2x+(\cos xy)\cfrac{d}{dx}(xy)\\
                            2y\cfrac{dy}{dx}&=2x+(\cos xy)(y+x\cfrac{dy}{dx})\\
                            (2y-x\cos xy)\cfrac{dy}{dx}&=2x+y\cos xy\\
                            \cfrac{dy}{dx}&=\cfrac{2x+y\cos xy}{2y-x\cos xy}
                        \end{aligned}
                    \end{equation}
        \subsection{Related Rates}
            \paragraph{Related Rate Equations}
                \[V=\cfrac{4}{3}\pi r^3\]
                \[\cfrac{dV}{dt}=\cfrac{dV}{dr}\cfrac{dr}{dt}=4\pi r^2\cfrac{dr}{dt}\]
            \textbf{Strategy}\\
                \textit{1. Draw a Picture and name the variables and constants.}\\
                \textit{2. Write down the numerical information.}\\
                \textit{3. Write down what you are asked to find.}\\
                \textit{4. Write an equation that relates the variables.}\\
                \textit{5. Differentiate with respect to t.}\\
                \textit{6. Evaluate.}\\
        \subsection{Linearization and Differentials}
            \paragraph{Linearization} If $f$ is a differentiable at $x=a$, then the approximating function
            \[L(x)=f(a)+f'(a)(x-a)\]
            is the \textbf{linearization} of $f$ at $a$. The approximation
            \[f(x)\approx L(x)\]
            of $f$ by $L$ is the \textbf{standard linear approximation} of $f$ at $a$. The point $x=a$ is the \textbf{center} of the approximation.
            \paragraph{Differentials} Let $y=f(x)$ be a differentiable function. The \textbf{differential} $\bm{dx}$ is an independent variable. The \textbf{differential} $\bm{dy}$ is
            \[dy=f'(x)dx\]
            \paragraph{Estimating with Differentials}
            Since
            \[f(a+dx)=f(a)+\Delta y\]
            the differential approximation gives
            \[f(a+dx)\approx f(a)+dy\]
            when $dx=\Delta x$. Thus the approximation $\Delta y \approx dy$ can be used to estimate $f(a+dx)$ when $f(a)$ is known, $dx$ is small,and $dy=f'(a)dx$.
            \paragraph{Error in Differential Approximation}
            If $y=f(x)$ is differentiable at $x=a$ and $x$ changes from $a$ to $a+\Delta x$, the change $\Delta y$ in $f$ is given by 
            \[\Delta y=f'(a) \Delta x+\epsilon\Delta x\]
            in which $\epsilon\to 0$ as $\Delta x\to 0$\\

            \paragraph{Sensitivity to Change}
            \begin{tabular}{ccc}
                \hline
                &\textbf{True}&\textbf{Estimated}\\
                \hline
                Absolute change & $\Delta f=f(a+dx)-f(a)$ & $df=f'(a)dx$\\
                Relative change&$\cfrac{\Delta f}{f(a)}$&$\cfrac{df}{f(a)}$\\
                Percentage change&$\cfrac{\Delta f}{f(a)}\times 100$&$\cfrac{df}{f(a)}\times 100$\\
            \end{tabular}

    \newpage
    \section{Applications of Derivatives}
        \subsection{Extreme Values of Functions}
            \paragraph{DEFINITIONS} Let $f$ be a function with domain $D$. Then $f$ has an \textbf{absolute maximum} value on $D$ at a point $c$ if
            \[f(x)\ge f(c)\quad \text{for all $x$ in $D$}\]
            and an \textbf{absolute minimum} value on $D$ at $c$ if
            \[f(x)\ge f(c)\quad \text{for all $x$ in $D$}\]
            which are called \textbf{extreme values} of the function $f$.
            \paragraph{Local(Relative) Extreme Values} A function $f$ has a $local maximum$ value at a point $c$ within its domain $D$ if $f(x) \le f(c)$ for all $x\in D$ lying in some open interval containing $c$.\\ \\
            A function $f$ has a \textbf{local minimum} value at a point $c$ within its domain $D$ if $f(x) \ge f(c)$ for all $x\in D$ lying in some open interval containing $c$.

            \paragraph{Finding Extrema}
            \subparagraph{Theorem---The First Derivative Theorem for Local Extreme Values} If $f$ has a local maximum or minimum value at an interior point $c$ of its domain, and if $f'$ is defined at $c$, then
            \[f'(c)=0\]
            \subparagraph{Definition} An interior point of the domain of a function $f$ where $f'$ is zero or undefined is \textbf{critical point} of $f$
        \subsection{The Mean Value Theorem*}
            \paragraph{Rolle's Theorem} Suppose that $y=f(x)$ is continuous over the closed interval $[a,b]$ and differentiable at every point of its interior $(a,b)$. If $f(a)=f(b)$, then there is at least one number $c$ in $(a,b)$ at which $f'(c)=0$.
            \paragraph{The Mean Value Theorem*} Suppose $y=f(x)$ is continuous over a closed interval $[a,b]$ and differentiable on the interval's interior $(a,b)$. Then there is at least one point $c$ in $(a,b)$ at which
            \[\cfrac{f(b)-f(a)}{b-a}=f'(c).\]
            \paragraph{Corollary 1} If $f'(x)=0$ at each point $x$ of an open interval $(a,b)$, then $f(x)=C$ for all $x\in (a,b)$, where $C$ is a constant.
            \paragraph{Corollary 2} If $f'(x)=g'(x)$ at each point $x$ in an open interval $(a,b)$, then there exists a constant $C$ such that $f(x)=g(x)+C$ for all $x\in (a,b)$. That is, $f-g$ is a constant function on $(a,b)$. 
        \subsection{Monotonic Functions and the First Derivative Test}
            \paragraph{Corollary 3} Suppose that $f$ is continuous on $[a,b]$ and differentiable on $(a,b)$.\\
                If $f'(x) > 0$ at each point $x\in (a,b)$, then $f$ is increasing on $[a,b]$.\\
                If $f'(x) < 0$ at each point $x\in (a,b)$, then $f$ is decreasing on $[a,b]$.\\
        \subsection{Concavity and Curve Sketching}
            \paragraph{Concavity} The graph of a differentiable function $y=f(x)$ is\\
            (a) \textbf{concave up} on an open interval $I$ if $f'$ is increasing on $I$.\\
            (b) \textbf{concave down} on an open interval $I$ if $f'$ is decreasing on $I$.

            \paragraph{The Second Derivative Test for Concavity} 
            Let $y=f(x)$ be twice-differentiable on an interval $I$.\\
            1. If $f''>0$ on $I$, the graph of $f$ over $I$ is concave up.\\
            2. If $f''<0$ on $I$,the graph of $f$ over $I$ is concave down.\\
            
            \paragraph{Points of Inflection}
                \subparagraph{Definition} A point $(c,f(c))$ where the graph of a function has a tangent line and where the Concavity changes is a \textbf{point of inflection}.
            \newline\\
            *At a point of inflection $(c,f(c))$, either $f''(c)=0$ or $f''y$\\
            \paragraph{Second Derivative Test for local Extrema}
            Suppose $f''$ is continuous on an open interval that contains $x=c$.\\
            1. If $f'(c)=0$ and $f''(c)<0$, then $f$ has a local maximum at $x=c$.\\
            2. If $f'(c)=0$ and $f''(c)>0$, then $f$ has a local minimum at $x=c$.\\
            3. If $f'(c)=0$ and $f''(c)=0$, then the test fails. The function $f$ may have a local maximum, a local minimum, or neither.\\

        \subsection{Applied Optimization}
            \textbf{Solving Applied Optimization Problems}\\
            1. \textit{Read the problem.}\\
            2. \textit{Draw a picture.}\\
            3. \textit{Introduce variables.}\\
            4. \textit{Write an equation for the unknown quantity.}\\
            5. \textit{Test the critical points and endpoints in the domain of the unknown.}\\
        
        \subsection{Newton's Method}
            \textbf{Newton's Method}\\
            1. Guess a first approximation to a solution of the equation $f(x)=0$. A graph of $y=f(x)$ may help.
            2. Use the first approximation to get a second, the second to get a third, and so on, using the formula
            \[x_{n+1}=x_n -\cfrac{f(x_n)}{f'(x_n)},\quad \text{if $f'(x_n)\ne 0$}.\]

        \subsection{Antiderivative}
            \paragraph{DEFINITION} A function $F$ is an \textbf{antiderivative} of $f$ on an interval $l$ if $F'(x)=f(x)$ for all $x$ in $I$.
            \paragraph{THEOREM} If $F$ is an antiderivative of $f$ on an interval $I$, then the most general antiderivative of $f$ on $I$ is
            \[F(x)+C\]
            where $C$ is an arbitrary constant.
            \\\\
            \textbf{Antiderivative formulas, $k$ a nonzero constant}\\
            \begin{tabular}{|ll|}
                \hline
                \textbf{Function}&\textbf{General antiderivative}\\
                \hline
                1. $x^n$&$\cfrac{1}{n+1}x^{n+1}+C,\quad n\ne -1$\\
                2. $\sin kx$&$-\cfrac{1}{k}\cos kx+C$\\
                3. $\cos kx$&$\cfrac{1}{k}\sin kx+C$\\
                4. $\sec ^2 kx$&$\cfrac{1}{k}\tan kx+C$\\
                5. $\csc ^2 kx$&$-\cfrac{1}{k}\cot kx+C$\\
                6. $\sec kx \tan kx$&$\cfrac{1}{k}\sec kx+C$\\
                7. $\csc kx \cot kx$&$-\cfrac{1}{k}\csc kx+C$\\&\\
                \hline
            \end{tabular}
            \\\\
            \textbf{Antiderivative linearity rules}\\
            \begin{tabular}{|lll|}
                \hline
                &\textbf{Function}&\textbf{General antiderivative}\\
                \hline956
                1.  \textit{Constant Multiple Rule:}&$kf(x)$&$kF(x)+C$, $k$ a constant\\
                2.  \textit{Negative Rule:}&$-f(x)$&$-F(x)+C$\\
                3.  \textit{Sum or Difference Rule:}&$f(x)\pm g(x)$&$F(x)\pm G(x)+C$\\
                \hline
            \end{tabular}
            \paragraph{Indefinite Integrals}
                \subparagraph{DEFINITION} The collection of all antiderivatives of $f$ is called the \textbf{indefinite integral} of $f$ with respect to $x$, and is denoted by
                \[\int f(x)\; dx\]
                The symbol $\int$ is an \textbf{integral sign}. The function $f$ is the integrand of the integral, and $x$ is the \textbf{variable of integration}.
            \paragraph{examples}
            \[\int 2x\; dx=x^2+C\]
            \[\int \cos x \; dx=\sin x+C\]
            \[\int (\sec ^2 x+\cfrac{1}{x\sqrt{x}})\; dx=\tan x+\sqrt{x}+C\]

    \newpage
    \section{Integrals}
        \subsection{Area and Estimating with Finite Sums}
            \[\text{SUM}=\sum_{i=1}^{n}f(c_{i})\Delta x\]
        \subsection{Sigma Notation and Limits of Finite Sums}
            \textbf{Finite Sums and Sigma Notation}\\\\
            \textbf{Sigma notation} enables us to write a sum with many terms in the compact form
            \[\textbf{Sigma Notation}:\quad\sum_{k=1}^{n}a_{k}=a_1+a_2+a_3+\cdots +a_{n-1}+a_{n}\]
            \textbf{Algebra Rules for Finite Sums}\\
            \begin{tabular}{|ll|}
                \hline
                1. \textit{Sum Rule:}&$\sum_{k=1}^{n}(a_k+b_k)+\sum_{k=1}^{n}a_k+\sum_{k=1}^{n}b_k$\\
                2. \textit{Difference Rule:}&$\sum_{k=1}^n(a_k-b_K)=\sum_{k=1}^{n}a_k-\sum_{k=1}^{n}b_k$\\
                3. \textit{Constant Multiple Rule:}&$\sum_{k=1}^{n}ca_k=c\cdot \sum_{k=1}^{n}a_k$\\
                4. \textit{Constant Value Rule:}&$\sum_{k=1}^{n}c=n\cdot c$\\
                \hline
            \end{tabular}
            \\
            \begin{tabular}{|ll|}
                \hline
                The first $n$ squares:&$\sum_{k=1}^{n}k^2=\cfrac{n(n+1)(2n+1)}{6}$\\
                The first $n$ cubes:&$\sum_{k=1}^{n}k^3=(\cfrac{n(n+1)}{2})^2$\\
                \hline
            \end{tabular}
            \paragraph{Riemann Sums}
                \subparagraph{Riemann sum for $f$ on the interval $\bf{[a,b]}$.}
                \[S_n=\sum_{k=1}^{n}f(c_k)\Delta x_k=\sum_{k=1}^{n}f(a+k\cfrac{(b-a)}{n})\cdot (\cfrac{b-a}{n})\]

        \subsection{The Definite Integral}
            \paragraph{Definition of the Definite Integral}
                Let $f(x)$ be a function defined on a closed interval $[a,b]$. We say that a number $J$ is the \textbf{definite integral of $f$ over $[a,b]$} and that $j$ is the limit of the Riemann sums $\sum_{k=1}^{n}f(c_k)\Delta x_k$ if the following condition is satisfied.\\
                Given any number $\epsilon > 0$ there is a corresponding number $\delta > 0$ such that for every partition $P={x_0,x_1,\cdots,x_n}$ of $[a,b]$ with $\Vert P \Vert < \delta$ and any choice of $c_k$ in $[x_{k-1},x_k]$, we have
                \[\lvert \sum_{k=1}^{n}f(c_k)\delta x_k -J\rvert<\epsilon\]
                \[J=\lim\limits_{\Vert P\Vert \to 0}\sum_{k=1}^{n}f(c_k)\Delta x_k\]
                \[\textbf{Symbol:}\quad\int_{a}^{b}f(x)dx\]
                named \textbf{``Integral of $f$ from $a$ to $b$''}.
                \[\int_{a}^{b}f(x)dx=\lim\limits_{n\to \infty}\sum_{k=1}^{n}f(a+k\cfrac{(b-a)}{n})\cdot (\cfrac{b-a}{n})\]
            \paragraph{Integrable and Nonintegrable Functions}
                \subparagraph{Integrability of Continuous Functions} If a function $f$ is continuous over the interval $[a,b]$, or if $f$ has at most finitely many jump discontinuities there, then the definite integral $\int_{a}^{b} f(x)dx$ exists and $f$ is integrable over $[a,b]$.
            \paragraph{Properties of Definite Integrals}
                \[\int_b^af(x)dx=-\int_a^bf(x)dx\]
                \[\int_a^af(x)dx=0\]
                \subparagraph{Theorem} When $f$ and $g$ are integrable over the interval $[a,b]$, the definite integral satisfies the rules below:\\\\
                \textbf{Rules satisfied by definite integrals}\\
                \begin{tabular}{|ll|}
                    \hline
                    1. \textit{Order of Integration:}&$\int_b^af(x)dx=-\int_a^bf(x)dx$\\
                    2. \textit{Zero Width Interval:}&$\int_a^af(x)dx=0$\\
                    3. \textit{Constant Multiple:}&$\int_a^bkf(x)dx=k\int_a^bf(x)dx$\\
                    4. \textit{Sum and Difference:}&$\int_a^b(f(x)\pm g(x))dx=\int_a^bf(x)dx\pm \int_a^bg(x)dx$\\
                    5. \textit{Additivity:}&$\int_a^bf(x)dx+\int_b^cf(x)dx=\int_a^cf(x)dx$\\
                    6. \textit{Max-Min Inequality:}&If $f$ has maximum value max $f$ and minimum\\& value min $f$ on $[a,b]$, then\\&$\text{min}f\cdot(b-a)\le \int_a^bf(x)dx\le \text{max}f\cdot(b-a)$\\
                    7. \textit{Domination:}&$f(x)\ge g(x)\ \text{on}\ [a,b] \Rightarrow \int_a^bf(x)dx\le \int_a^b g(x)dx$\\&$f(x)\ge 0\ \text{on}\ \Rightarrow\int_a^bf(x)dx\ge 0$\\
                    \hline
                \end{tabular}
            \paragraph{Area Under the Graph of a Nonnegative Funcion}
                \subparagraph{Definition} If $y=f(x)$ is nonnegative and integrable over a closed interval $[a,b]$, then the \textbf{area under the curve $\bf{y=f(x)}$ over $\bf{[a,b]}$} is the integral of $f$ from $a$ to $b$.
                \[A=\int_a^bf(x)dx\]
                We have the following rules:
                \begin{equation}
                    \begin{aligned}
                        \int_a^bxdx&=\cfrac{b^2}{2}-\cfrac{a^2}{2},\quad a<b\\
                        \int_a^bcdx &=c(b-a),\quad c\text{ any constant}\\
                        \int_a^bx^2dx&=\cfrac{b^3}{3}-\cfrac{a^3}{3},\quad a<b\\
                    \end{aligned}
                \end{equation}
            \paragraph{Average Value of a Continuous Function Revisited}
                \[\text{av}(f)=\cfrac{1}{b-a}\int_a^bf(x)dx\]
                is $f$'s \textbf{average value on $\bf{[a,b]}$}, also called it's \textbf{mean}.
        \subsection{The Fundamental Theorem of Calculus}
            \paragraph{Mean Value Theorem for Definite Integrals} If $f$ is continuous on $[a,b]$, then at some point $c$ in $[a,b]$,
            \[f(c)=\cfrac{1}{b-a}\int_a^b f(x)dx\]
            \paragraph{Fundamental Theorem, Part 1}
            \text{}\\
            \par If $f(t)$ is an integrable function over a finite interval $I$, then the integral from any fixed number $a\in I$ to another number $x\in I$ defines a new function $F$ whose value at $x$ is
            \[F(x)=\int_a^xf(t)dt\]
            \par For example, if $f$ is nonnegative and $x$ lies to the right of $a$, then $F(x)$ is the area under the graph from $a$ to $x$. The variable x is the upper limit of integration of an integral, but $F$ is just like any other real-valued function of a real variable. For each value of the input $x$, there is a well-defined numerical output, in this case the definite integral of $f$ from $a$ to $x$.
            \par This equation gives a way to define new functions, but its importance now is the connection it makes between integrals and derivatives. If $f$ is any continuous function, then the Fundamental Theorem asserts that $F$ is a differentiable function of $x$ whose derivative is $f$ itself. At every value of $x$, it asserts that
            \[\cfrac{d}{dx}F(x)=f(x).\]
            \par To gain some insight into why this result holds, we look at the geometry behind it.
            \par If $f\ge 0$ on $[a,b]$, then the computation of $F'(x)$ from the definition of teh derivative means taking the limit as $h\to 0$ of the difference quotient
            \[\cfrac{F(x+h)-F(x)}{h}\]
            \par For $h>0$, the numerator is obtained by subtracting two areas, so it is the area under the graph of $f$ from $x$ to $x+h$. If $h$ is small, this area is approximately equal to the area of the rectangle of height $f(x)$ and width $h$. That is
            \[F(x+h)-F(x)\approx hf(x)\]
            \par Dividing both sides of this approximation by $h$ and letting $h\to 0$, it is reasonable to expect that
            \[F'(x)=\lim\limits_{h\to 0}\cfrac{F(x+h)-F(x)}{h}=f(x)\]
            \par This result is true even if the function $f$ is not positive, and it forms the first part of the Fundamental Theorem of Calculus.
            \textbf{EXAMPLE:Find $dy/dx$ if}\\
            \[y=\int_{1+3x^2}^4 \cfrac{1}{2+t}dt\]
            \[\text{let}\ u=1+3x^2\]
            \begin{equation}
                \begin{aligned}
                    \cfrac{dy}{dx}&=\cfrac{dy}{du}\cdot\cfrac{du}{dx}\\
                    &=\cfrac{d}{du}\int_u^4 \cfrac{1}{2+t}dt\cdot(6x)\\
                    &=-\cfrac{d}{du}\int_4^u\cfrac{1}{2+t}dt\cdot(6x)\\
                    &=-\cfrac{1}{2+u}\cdot(6x)\\
                    &=-\cfrac{2x}{x^2+1}
                \end{aligned}
            \end{equation}
            \paragraph{Proof of Theorem}
            \begin{equation}
                \begin{aligned}
                    F'(x)&=\lim\limits_{h\to 0}\cfrac{F(x+h)-F(x)}{h}\\
                    &=\lim\limits_{h\to 0}\cfrac{1}{h}[\int_0^{x+h}f(t)dt-\int_a^x f(t)dt]\\
                    &=\lim\limits_{h\to 0}\cfrac{1}{h}\int_x^{x+h}f(t)dt\\
                \end{aligned}
            \end{equation}
            \par According to the Mean Value Theorem for Definite Integrals, the value before taking the limit in the last expression is one of the values taken on by $f$ in the interval between $x$ and $x+h$. That is, for some numer $c$ in this interval,\\
            \begin{equation}
                \cfrac{1}{h}\int_x^{x+h}f(t)dt=f(c)
            \end{equation}
            \par As $h\to 0$, $x+h$ approaches $x$, forcing $c$ to approach $x$ also (because $c$ is trapped between $x$ and $x+h$). Since $f$ is continuous at $x$, f(c) approaches $f(x)$
            \begin{equation}
                \lim\limits_{h\to 0}f(c)=f(x)
            \end{equation}
            \par In conclusion, we have
            \begin{equation}
                \begin{aligned}
                    F'(x)&=\lim\limits_{h\to 0}\cfrac{1}{h}\int_x^{x+h}f(t)dt\\
                    &=\lim\limits_{h\to 0}f(c)\\
                    &=f(x)\\
                \end{aligned}
            \end{equation}
            \par If $x=a$ or $b$, then the limit of Equation is interpreted as a one-sided limit with $h\to 0^+$ or $h\to 0^-$, respectively.
            \paragraph{Fundamental Theorem, Part 2(The Evaluation Theorem)}
                \subparagraph{The Fundamental Theorem of Calculus, Part 2}
                If $f$ is continuous over $[a,b]$ and $F$ is any antiderivative of $f$ on $[a,b]$, then
                \[\int_a^b f(x)dx=F(b)-F(a)=F(x)\Bigg]_a^b=\Bigg[F(x)\Bigg]_a^b\]
                \subparagraph{Example:}
                \begin{equation}
                    \begin{aligned}
                        \int_1^4(\cfrac{3}{2}\sqrt{x}-\cfrac{4}{x^2})dx&=\Bigg[x^{3/2}+\cfrac{4}{x}\Bigg]_1^4\\
                        &=[8+1]-[5]\\
                        &=4
                    \end{aligned}
                \end{equation}
            \paragraph{The Integral of a Rate}
            \[\int_a^bF'(x)dx=F(b)-F(a)\]
            \[F(b)=F(a)+\int_a^b F'(x)dx\]
            \paragraph{The Relationship Between Integration and Differrentiation}
            \[\cfrac{d}{dx}\int_a^xf(t)dt=f(x)\]
        \subsection{Indefinite Integrals and the Substitution Method}
            The indefinite integral $\int$ notation  means for any antiderivative $F$ of $f$,
            \[\int f(x)dx=F(x)+C\]
            where $C$ is an arbitrary constant.
            \paragraph{Substitution: Running the Chain Rule Backwards}
                If $u$ is a differentiable function of $x$ and $n$ is an number different form -1, the Chain Rule tells us that 
                \[\cfrac{d}{dx}(\cfrac{u^{n+1}}{n+1})=u^n\cfrac{du}{dx}\]
                From another point of view, this same equation says that $u^{n+1}/(n+1)$ is one of the antiderivatives of the function $u^n(du/dx)$. Therefore,
                \[\int u^n\cfrac{du}{dx}dx=\cfrac{u^{n+1}}{n+1}+C\]
                The integral in the equation is equal to the simpler integral
                \[\int u^n du=\cfrac{u^{n+1}}{n+1}+C\]
                \[du=\cfrac{du}{dx}dx\]
            \paragraph{Example 1} Find the integral $\int (x^3+x)^5 (3x^2+1)dx$
                \subparagraph{Solution} We set $u=x^3+x$. Then
                \[du=\cfrac{du}{dx}dx=(3x^2+1)dx\]
                so that by Substitution we have
                \begin{equation}
                    \begin{aligned}
                        \int (x^3+x)^5(3x^2+1)dx&=\int u^5 du\\
                        &=\cfrac{u^6}{6}+C\\
                        &=\cfrac{(x^3+x)^6}{6}+C\\
                    \end{aligned}
                \end{equation}
            \paragraph{Example 2} Find $\int \sqrt{2x+1}dx$
                \subparagraph{Solution} 
                \begin{equation}
                    \begin{aligned}
                        \int \sqrt{2x+1}dx&=\cfrac{1}{2}\int \sqrt{2x+1}\cdot 2dx\\
                        &=\cfrac{1}{2}\int u^{\cfrac{1}{2}}du\\
                        &=\cfrac{1}{2}\cfrac{u^{3/2}}{3/2}+C\\
                        &=\cfrac{1}{2}(2x+1)^{\frac{3}{2}}+C
                    \end{aligned}
                \end{equation}
            \paragraph{The Substitution Rule} If $u=g(x)$ is a differentiable function whose range is an interval $I$, and $f$ is continuous on $I$, then
            \[\int f(g(x))g'(x)dx=\int f(u)du\]
            \paragraph{Example 3} Find $\int \sec^2(5x+1)\cdot 5dx$
                \subparagraph{Solution} We substitute $u=5x+1$ and $du=5dx$. Then
                \begin{equation}
                    \begin{aligned}
                        \int \sec^2(5x+1)\cdot 5 dx&=\int \sec ^2u\ du\\
                        &=\tan u+C\\
                        &=\tan(5x+1)+C
                    \end{aligned}
                \end{equation}
        \subsection{Definite Integral Substitutions and the Area Between Curves}
            \paragraph{The Substitution Formula}
                \subparagraph{Substitution in Definite Integrals} If $g'$ is continuous on the interval $[a,b]$ and $f$ is continuous on the range of $g(x)=u$, then
                \[\int_a^bf(g(x))\cdot g'(x)dx=\int_g(a)^g(b)f(u)du\]
                \subparagraph{Proof} Let $F$ denote any antiderivative of $f$. Then,
                \begin{equation}
                    \begin{aligned}
                        \int_a^bf(g(x))\cdot g'(x)dx&=F(g(x))\Bigg ]_{x=a}^{x=b}\\
                        &=F(g(b))-F(g(a))\\
                        &=F(u)\Bigg ]_{u=g(a)}^{u=g(b)}\\
                        &=\int_{g(a)}^{g(b)}f(u)du\\
                    \end{aligned}
                \end{equation}
            \paragraph{Example 1} Evaluate $\int_{-1}^1 3x^2\sqrt{x^3+1}dx$
                \subparagraph{Solution}
                \begin{equation}
                    \begin{aligned}
                        \int_{-1}^1 3x^2\sqrt{x^3+1}dx&=\int_0^2\sqrt{u}du\\
                        &=\cfrac{2}{3}u^{3/2}\Bigg ]_0^2\\
                        &=\cfrac{2}{3}[2^{3/2}-0^{3/2}]\\
                        &=\cfrac{2}{3}[2\sqrt{2}]\\
                        &=\cfrac{4\sqrt{2}}{3}
                    \end{aligned}
                \end{equation}\\
            \paragraph{Definite Integrals of Symmetric Functions}
                \subparagraph{Theorem} Let $f$ be continuous on the symmetric interval $[-a,a]$.\\
                (a) If $f$ is even, then $\int_{-a}^af(x)dx=2\int_0^af(x)dx$\\
                (b) If $f$ is odd, then $\int_{-a}^af(x)dx=0$
            \paragraph{Areas Between Curves}
                \subparagraph{Definition} If $f$ and $g$ are continuous with $f(x)\ge g(x)$ throughout $[a,b]$, then the \textbf{area of the region between the curves $\bf{y=f(x)}$ and $\bf{y=g(x)}$ from $\bf(a)$ to $\bf{b}$} is the integral of $(f-g)$ from $a$ to $b$:
                \[A=\int_a^b[f(x)-g(x)]dx\]
            \paragraph{Integration with Respect to $y$}
                \[A=\int_c^d [f(y)-g(y)]dy\]
    
    \newpage
    \section{Applications of Definite Integrals}
        \subsection{Volumes Using Cross-Sections}
            \paragraph{Definition} The \textbf{volume} of a solid of integrable cross-sectional area $A(x)$ from $x=a$ to $x=b$ is the integral of $A$ from $a$ to $b$.
            \[V=\int_a^b A(x)dx\]
                \subparagraph{Calculating the Volume of a Solid}
                \text{}\\
                1. \textit{Sketch the solid and a typical cross-section.}\\
                2. \textit{Find a formula for $A(x)$, the area of a typical cross-section.}\\
                3. \textit{Find the limits of integration.}\\
                4. \textit{Integrate $A(x)$ to find the volume.}
            \paragraph{Solid of Revolution: The Disk Method} The solid generated by rotating (or revolving) a plane region about an axis in its plane is called a \textbf{solid of revolution}.
            \[A(x)=\pi (\text{radius})^2=\pi [R(x)]^2\]
                \subparagraph{Volume by Disks for Rotation About the $x$-axis}
                \[V=\int_a^b A(x)\ dx=\int_a^b\pi [R(x)]^2\ dx\]
            \paragraph{Example} Find the volume of the solid generated by revolving the region bounded by $y=\sqrt{x}$ and the lines $y=1$, $x=4$ about the line $y=1$.
                \subparagraph{Solution}
                \begin{equation}
                    \begin{aligned}
                        V&=\int_1^4 \pi[R(x)]^2\ dx\\
                        &=\int_1^4 \pi[\sqrt{x}-1]^2\ dx\\
                        &=\pi \int_1^4[x-2\sqrt{x}+1] dx\\
                        &=\pi \Bigg [\cfrac{x^2}{2}-2\cdot\cfrac{2}{3}x^{3/2}+x\Bigg]_1^4\\
                        &=\cfrac{7\pi}{6}
                    \end{aligned}
                \end{equation}
                \subparagraph{Volume by Disks for Rotation About the $y$-axis}
                \[V=\int_c^d A(y)\ dy=\int_c^d\pi [R(y)]^2\ dy\]
            \paragraph{Solids of Revolution: The Washer Method}
                \subparagraph{Volume by Washers for Rotation About the $x$-axis}
                \[V=\int_a^b A(x)\ dx=\int_a^b \pi([R(x)]^2-[r(x)]^2)dx\]
        \subsection{Volumes Using Cylindrical Shells}
            \paragraph{Slicing with Cylinders}
            Unrolling a cylindrical shell shows that its volume is approximately that of a rectangular slab with area $A(x)$ and thickness $\Delta x$.
            \paragraph{The Shell Method}
            \begin{equation}
                \begin{aligned}
                    \Delta V_k&=2\pi \times \text{average shell radius} \times \text{shell height} \times \text{thickness}\\
                    &=2\pi \cdot(c_k-L) \cdot f(c_k)\cdot \Delta x_k\\
                    V&=\lim\limits_{n\to \infty} \sum_{k=1}^n \Delta V_k\\
                    &=\int_a^b 2\pi \text{(shell radius)(shell height)} dx\\
                    &=\int_a^b 2\pi (x-L)f(x)\ dx
                \end{aligned}
            \end{equation}
        \subsection{Arc Length}
            \paragraph{Length of a Curve $y=f(x)$}
                Suppose the curve whose length we want to find is the graph of the function $y=f(x)$ from $x=a$ to $x=b$. In order to derive an integral formula for the length of the curve, we assume that $f$ has a continuous derivative at every point of every point of $[a,b]$. Such a function is called \textbf{smooth}, and its graph is a \textbf{smooth curve} because it does not have any breaks, corners, or cusps.
                \[L_k=\sqrt{(\Delta x_k)^2+(\Delta y_k)^2}\]
                \[\sum_{k=1}^n L_k=\sum_{k=1}^n \sqrt{(\Delta x_k)^2+(\Delta y_k)^2}\]
                \[\Delta y_k=f'(c_k)\Delta x_k\]
                \[\sum_{k=1}^n L_k=\sum_{k=1}^n \sqrt{(\Delta x_k)^2+(f'(c_k)\Delta x_k)^2}=\sum_{k=1}^n \sqrt{1+[f'(c_k)]^2}\Delta x_k\]
                \[\lim\limits_{n\to \infty}\sum_{k=1}^n L_k=\lim\limits_{n\to \infty}\sum_{k=1}^n \sqrt{1+[f'(c_k)]^2}\Delta x_k=\int_a^b \sqrt{1+[f'(x)]^2}dx\]
                \[L=\int_a^b \sqrt{1+[f'(x)]^2}dx=\int_a^b \sqrt{1+(\cfrac{dy}{dx})^2}dx\]
            \paragraph{Dealing with Discontinuities in $dy/dx$}
                \subparagraph{Formula for the Length of $x=g(y),\ c\le y \le d$}
                \[L=\int_c^d \sqrt{1+[g'(y)]^2}dy=\int_c^d \sqrt{1+(\cfrac{dx}{dy})^2}dy\]
            \paragraph{The Differential Formula for Arc Length}
                \text{}\\
                \par If $y=f(x)$ and if $f'$ is continuous on $[a,b]$, then by the Fundamental Theorem of Calculus we can define a new function
                \[s(x)=\int_a^x \sqrt{1+[f'(t)]^2}\ dt\]
                \[\cfrac{ds}{dx}=\sqrt{1+[f'(x)]^2}=\sqrt{1+\bigg (\cfrac{dy}{dx}\bigg )^2}\]
                \[ds=\sqrt{1+\bigg(\cfrac{dy}{dx}\bigg)^2}\ dx\]
                \[ds=\sqrt{dx^2+dy^2}\]
        \subsection{Areas of Surfaces of Revolution}
            \paragraph{Defining Surface Area}
            \begin{equation}
                \begin{aligned}
                    \text{Frustum surface area}&=2\pi \cdot \cfrac{f(x_{k-1})+f(x_k)}{2}\cdot \sqrt{(\Delta x_k)^2+(\Delta y_k)^2}\\
                    &=\pi (f(x_{k-1})+f(x_k))\sqrt{(\Delta x_k)^2+(\Delta y_k)^2}
                \end{aligned}
            \end{equation}
                \subparagraph{Definition} If the function $f(x)\ge 0$ is continuously differentiable on $[a,b]$, the \textbf{area of the surface} generated by revolving the graph of $y=f(x)$ about the $x$-axis is  
                \[S=\int_a^b 2\pi y\sqrt{1+\bigg( \cfrac{dy}{dx} \bigg)^2} dx=\int_a^b 2\pi f(x) \sqrt{1+(f'(x))^2}\ dx\]
            \paragraph{Revolution About the $y$-Axis}
                \[S=\int_c^d 2\pi x\sqrt{1+\bigg( \cfrac{dx}{dy} \bigg)^2} dy=\int_c^d 2\pi g(y) \sqrt{1+(g'(x))^2}\ dy\]
        \subsection{Work and Wfuid Forces}
            \paragraph{Work Done by a Constant Force}
            \[W=Fd\qquad\text{(Constant-force formula for work.)}\]
            \paragraph{Work Done by a Variable Force Along a Line}
            \[W=\int_a^bF(x)dx\]
            \paragraph{Hooke's Law for Springs:$F=kx$}
                \subparagraph{Hooke's Law} The force required to hold a stretched or compressed spring $x$ units from its nature (unstressed) length is proportional to $x$. In symbol
                \[F=kx\]
            \paragraph{Lifting Objects and Pumping Liquidsfrom Containers}
            \paragraph{Fluid Pressure and Forces}    
                \subparagraph{The Pressure-Depth Equation}
                \[p==wh\]
                \[F=pA=whA\]
                \subparagraph{The Integral for Fluid Force Against a Vertical Flat Plate}
                \[F=\int_a^b w\cdot(\text{strip depth})\cdot L(y)\ dy\]
        \subsection{Moments and Centers of Mass}
            \paragraph{Masses Along a Line}
            \[\textbf{System Torque}=\sum_{k=1}^n m_k g x_k\]
            \[M_0=\text{Moment of system about origin}=\sum_{k=1}^n m_k x_k\]
            \begin{equation}
                \begin{aligned}
                    \sum (x_k-\overline{x})m_k g&=0\\
                    \overline{x}&=\cfrac{\sum m_k x_k}{\sum m_k}\\
                    &=\cfrac{\text{System moment about origin}}{\text{System mass}}
                \end{aligned}
            \end{equation}
            \paragraph{Masses Distributed over a Plane Region}
            \text{}\\
            \par System mass:   $M=\sum m_k$\\
            \par Moment about $x$-axis:     $M_x=\sum m_k y_k$\\
            \par Moment about $y$-axis:     $M_y=\sum m_k x_k$\\
            \[\overline{x}=\cfrac{M_y}{M}=\cfrac{\sum m_k x_k}{\sum m_k}\]
            \[\overline{y}=\cfrac{M_x}{M}=\cfrac{\sum m_k y_k}{\sum m_k}\]
            \paragraph{Thin, Flat Plates}
                \subparagraph{Moments, Mass and Center of Mass of a Thin Plate Covering a Region in the $xy$-Plane}
                \text{}\\ \par
                \begin{tabular}{|rl|}
                    \hline
                    Moment about the $x$-axis:&\quad $M_x=\int \widetilde{y} dm$\\
                    Moment about the $y$-axis:&\quad $M_y=\int \widetilde{x} dm$\\
                    Mass:&\quad $M=\int dm$\\
                    Center of mass:&\quad $\overline{x}=\cfrac{M_y}{M}$,\ $\overline{y}=\cfrac{M_x}{M}$\\
                    \hline
                \end{tabular}
            \paragraph{Plates Bounded by Two Curves}
                \[\overline{x}=\cfrac{1}{M}\int_a^b \delta x[f(x)-g(x)]dx\]
                \[\overline{y}=\cfrac{1}{M}\int_a^b \frac{\delta}{2}[f^2(x)-g^2(x)]dx\]
            \paragraph{Centroids}
            \paragraph{Fluid Forces and Centroids}
            \[F=w\overline{h}A\]
            \paragraph{The Theorems of Pappus}
                \subparagraph{Pappus's Theorem for Volumes}If  a  plane  region  is  revolved  once about a line in the plane that does not cut through the region’s interior, then the volume of the solid it generates is equal to the region’s area times the distance traveled by the region’s centroid during the revolution. If $\rho$ is the distance from the axis of revolution to the centroid, then
                \subparagraph{Proof} 
                \[V=\int_c^d 2\pi (\text{shell radius})(\text{shell height})\ dy=2\pi \int_c^d yL(y) dy\]
                \[\overline{y}=\cfrac{\int_c^d \widetilde{y}\ dA}{A}=\cfrac{\int_c^d yL(y)\ dy}{A}\]
                \[\int_{c}^{d} yL(y)\ dy =A \overline{y}\]
                \subparagraph{Pappus's Theorem for Surface Areas}  If  an  arc  of  a  smooth  plane curve is revolved once about a line in the plane that does not cut through the  arc’s  interior,  then  the  area  of  the  surface  generated  by  the  arc  equals  the  length $L$  of  the  arc  times  the  distance  traveled  by  the  arc’s  centroid  during  the  revolution. If $\rho$ is the distance from the axis of revolution to the centroid, then
                \[S=2\pi \rho L\]
    
    \newpage
    \section{Transcendental Functions} 
        \subsection{Inverse Functions and Their Derivatives}
            \paragraph{One-to-One Functions}
            A function $f(x)$ is \textbf{One-to-One} on a domain $D$ if $f(x_1)\ne f(x_2)$ whenever $x_1\ne x_2$ in $D$.
            \paragraph{Inverse Functions}
            Suppose that $f$ is a one-to-one function on a domain $D$ with range $R$. The \textbf{Inverse function} $f^{-1}$ is defined by
            \[f^{-1}(b)=a\quad \text{if}\quad f(a)=b\] 
            The domain of $f^{-1}$ is $R$ and the range of $f^{-1}$ is $D$.
            \begin{equation}
                \begin{aligned}
                    (f^{-1}\circ f)(x)=x&,\quad \text{for all $x$ in the domain of $f$}\\
                    (f\circ f^{-1})(y)=x&,\quad \text{for all $y$ in the domain of $f^{-1}$}\\
                \end{aligned}
            \end{equation}
            \paragraph{Finding Inverses}
            \paragraph{Derivatives of Inverses of Differentiable Functions}
                \subparagraph{The Derivative Rule for Inverses} If $f$ has an interval $I$ as domain and $f'(x)$ exists and is never zero on $I$, then $f^{-1}$ is differentiable at every point in its domain (the range of $f$). The value of $(f^{-1})'$ at a point $b$ in the domain of $f^{-1}$ is the reciprocal of the value of $f'$ at the point $a=f^{-1}(b)$:
                \[(f^{-1})'(b)=\cfrac{1}{f'(f^{-1}(b))}\]
                or
                \[\cfrac{df^{-1}}{dx}\Bigg|_{x=b}=\cfrac{1}{\cfrac{df}{dx}\Bigg|_{x=f^{-1}(b)}}\]
                \subparagraph{Proof}
                \begin{equation}
                    \begin{aligned}
                        f(f^{-1}(x))&=x\\
                        \cfrac{d}{dx}f(f^{-1}(x))&=1\\
                        \cfrac{d}{dx}f^{-1}(x)\cdot f'(f^{-1}(x))&=1\\
                        \cfrac{d}{dx}f^{-1}(x)&=\cfrac{1}{f'(f^{-1}(x))}
                    \end{aligned}
                \end{equation}
        \subsection{Natural Logarithms}
            \paragraph{Definition of the Natural Logarithm Function}
                The \textbf{natural logarithm} is the function given by
                \[\ln x=\int_1^x \cfrac{1}{t}dt,\quad x>0\]
                so $\ln 1=\int_1^1 \frac{1}{t}dt=0$
                \subparagraph{Definition} The \textbf{number $\bf{e}$} is that number in the domain of the natural logarithm satisfying
                    \[\ln (e)=\int_1^e\frac{1}{t}dt=1\]
                    \[e\approx 2.71828\]
            \paragraph{The Derivative of $y=\ln x$}
                \[\cfrac{d}{dx}\ln x=\cfrac{d}{dx}\int_1^x\cfrac{1}{t}dt=\cfrac{1}{x}\]
                \[\cfrac{d}{dx}\ln x=\cfrac{1}{x}\]
                \begin{equation}
                    \cfrac{d}{dx} \ln u=\cfrac{1}{u}\cdot \cfrac{du}{dx},\quad u>0
                \end{equation}
            \paragraph{Properties of Logarithms}
                    \subparagraph{Algebraic Properties of the Nature Logarithm} For any numbers $b>0$ and $x>0$, the natural logarithm satisfies the following rules:\\\par
                    \begin{tabular}{|ll|}
                        \hline
                        1. \textit{Product Rule:}&$\ln bx=\ln b+\ln x$\\
                        2. \textit{Quotient Rule:}&$\ln \cfrac{b}{x}=\ln b-\ln x$\\
                        3. \textit{Reciprocal Rule:}&$\ln \cfrac{1}{x}=-\ln x$\\
                        4. \textit{Power Rule:}&$\ln x^r=r\ln x$\\
                        \hline
                    \end{tabular}
            \paragraph{The Graph and Range of $\ln x$}
            \paragraph{The Integral $\int(1/u)du$}
                \[\int \cfrac{1}{u}du=\ln |u|+C\]
                If $u=f(x)$, then $du=f'(x)$ and
                \[\int \cfrac{f'(x)}{f(x)}dx=\ln |f(x)|+C\]
                whenever $f(x)$ is a differentiable function that is never zero.
            \paragraph{The Integrals of $\tan x$, $\cot x$, $\sec x$, and $\csc x$}
                \begin{equation}
                    \begin{aligned}
                        \int \tan x\ dx&=\int \cfrac{\sin x}{\cos x} dx=\int \cfrac{-du}{u}\\
                        &=-\ln |u| +C=-\ln |\cos x| +C\\
                        &=\ln \cfrac{1}{|\cos x|} +C=\ln |\sec x|+C\\
                        \int \cot x\ dx&=\int \cfrac{\cos x dx}{\sin x}=\int\cfrac{du}{u}\\
                        &=\ln |u|+C=\ln |\sin x|+C=-\ln |\csc x|+C\\
                        \int \sec x\ dx&=\int \sec x \cfrac{(\sec x+\tan x)}{(\sec x+\tan x)} dx\\
                        &=\int \cfrac{\sec ^2 x+\sec x\tan x}{\sec x+\tan x}dx\\
                        &=\int \cfrac{du}{u}=\ln |u|+C\\
                        &=\ln |\sec x+\tan x|+C\\
                        \int \csc x\ dx&=\int \csc x\cfrac{(\csc x+\cot x)}{(\csc x+\cot x)} dx\\
                        &=\int\cfrac{\csc^2 x+\csc x\cot x}{\csc x+\cot x}dx\\
                        &=\int \cfrac{-du}{u}=-\ln |u|+C\\
                        &=-\ln |\csc x+\cot x|+C\\
                    \end{aligned}
                \end{equation}
            \paragraph{Logarithmic Differentiation}
            Use laws of logarithms to simplify the formulas before differentiating.
        \subsection{Exponential Function}
            \paragraph{The Inverse of $\ln x$ and the Number $e$}
                \subparagraph{Definition} For every real number $x$, we define the \textbf{natural exponential function} to be $e^x=\exp x$.
                \subparagraph{Inverse Equations for $e^x$ and $\ln x$}
                \begin{equation}
                    \begin{aligned}
                        e^{\ln x}=x&\qquad (\text{all } x>0)\\
                        \ln(e^x)=x&\qquad (\text{all } x)\\
                    \end{aligned}
                \end{equation}
            \paragraph{The Derivative and Integral of $e^x$}
            \begin{equation}
                \begin{aligned}
                    \ln (e^x)&=x\\
                    \cfrac{d}{dx}\ln (e^x)&=1\\
                    \cfrac{1}{e^x}\cdot\cfrac{d}{dx}(e^x)&=1\\
                    \cfrac{d}{dx}e^x&=e^x\\
                    \cfrac{d}{dx}e^u&=e^u\cfrac{du}{dx}\\
                    \int e^u du&=e^u+C\\
                \end{aligned}
            \end{equation}
            \[\lim\limits_{x\to -\infty} e^x=0\]
            \[\lim\limits_{x\to \infty} e^x=\infty\]
            \paragraph{Laws of Exponents}
                \[e^{x_1}\cdot e^{x_2}=e^{x_1+x_2}\]
                \[e^{-x}=\cfrac{1}{e^x}\]
                \[\cfrac{e^{x_1}}{e^{x_2}}=e^{x_1-x_2}\]
                \[(e^{x_1})^r=e^{rx_1},\quad \text{if $f$ is rational}\]
            \paragraph{The General Exponential Function $a^x$}
                \subparagraph{Definition} For any number $a>0$ and $x$, the \textbf{exponential function with base }$\bf{a}$ is
                \[a^x=e^{x\ln a}\]
            \paragraph{Proof of the Power Rule(General Version)}
                \subparagraph{Definition} For any $x>0$ and for any real number $n$,
                \[x^n=e^{n\ln x}\]
                \subparagraph{General Power Rule for Derivatives}
                \text{}\\
                For $x>0$ and any real number $n$.
                \[\cfrac{d}{dx}x^n=nx^{n-1}\]
                If $x<0$, then the formula holds whenever the derivative, $x^n$, and $x^{n-1}$ all exist.
                \subparagraph{Proof} Differentiating $x^n$ with respect to $x$ gives\\
                \begin{equation}
                    \begin{aligned}
                        \cfrac{d}{dx}x^n&=\cfrac{d}{dx}e^{n\ln x}\\
                        &=e^{n\ln x}\cdot\cfrac{d}{dx}(n\ln x)\\
                        &=x^n\cdot \cfrac{n}{x}\\
                        &=nx^{n-1}
                    \end{aligned}
                \end{equation}
                \subparagraph{Example} Differentiate $f(x)=x^x,x>0$
                \begin{equation}
                    \begin{aligned}
                        f'(x)&=\cfrac{d}{dx}(e^{x\ln x})\\
                        &=e^{x\ln x}\cfrac{d}{dx}(x\ln x)\\
                        &=e^{x\ln x}\cdot(\ln x+x\cdot\cfrac{1}{x})\\
                        &=x^x(\ln x+1)
                    \end{aligned}
                \end{equation}
            \paragraph{The Number $e$ Expressed as a Limit}
                \subparagraph{The Number $e$ as a Limit}
                    \[e=\lim\limits_{x\to 0}(1+x)^{1/x}\]
                \subparagraph{Proof} If $f(x)=\ln x$, then $f'(x)=1/x$, so $f'(1)=1$. But, by the definition of derivative.
                    \begin{equation}
                        \begin{aligned}
                            f'(1)&=\lim\limits_{h\to 0}\cfrac{f(1+h)-f(1)}{h}\\
                            &=\lim\limits_{x\to 0}\cfrac{f(1+x)-f(1)}{x}\\
                            &=\lim\limits_{x\to 0}\cfrac{\ln (1+x)-\ln 1}{x}\\
                            &=\lim\limits_{x\to 0}\cfrac{1}{x}\ln (1+x)\\
                            &=\lim\limits_{x\to 0}\ln(1+x)^{1/x}\\
                            &=\ln \bigg[\lim\limits_{x\to 0} (1+x)^{1/x}\bigg]=1\\
                        \end{aligned}
                    \end{equation}
                    \[\lim\limits_{x\to 0}(1+x)^{1/x}=e\]
            \paragraph{The Derivative of $a^u$}
                \[\cfrac{d}{dx}a^x=\cfrac{d}{dx}e^{x\ln a}=e^{x\ln a}\cdot\cfrac{d}{dx}(x\ln a)=a^x\ln a\]
                \[\cfrac{d}{dx}a^u=a^u \ln a\cfrac{du}{dx}\]
                \[\int a^u du=\cfrac{a^u}{\ln a}+C\]
                \[\cfrac{d^2}{dx^2}(a^x)=\cfrac{d}{dx}(a^x\ln a)=(\ln a)^2 a^x\]
            \paragraph{Logarithms with Base $a$}
                \subparagraph{Definition} For any positive number $a\ne 1$, $\log_a x$ is the inverse function of $a^x$.
                \[a^{\log_a x}=x\quad(x>0)\]
                \[log_a(a^x)=x\quad (\text{all } x)\]
                \[\log_a x=\cfrac{\ln x}{\ln a}\]
                \[\ln xy=\ln x+\ln y\]
                \[\cfrac{\ln xy}{\ln a}=\cfrac{\ln x}{\ln a}+\cfrac{\ln y}{\ln a}\]
                \[\log_a xy=\log_a x+\log_a y\]
            \paragraph{Derivatives and Integrals Involving $log_a x$}
                \[\cfrac{d}{dx}(\log_a u)=\cfrac{d}{dx}\bigg(\cfrac{\ln u}{\ln a}\bigg)=\cfrac{1}{\ln a}\cfrac{d}{dx}(\ln u)=\cfrac{1}{\ln a}\cdot\cfrac{1}{u}\cfrac{du}{dx}\]
        \subsection{Exponential Change and Separable Differential Equations}
            \paragraph{Exponential Change}
                \[\cfrac{dy}{dt}=ky,\qquad y(0)=y_0\]
                \[y=y_0 e^{kt}\]
            \paragraph{Separable Differential Equations}
                \[\cfrac{dy}{dx}=f(x,y)\]
                \[\cfrac{d}{dx} y(x)=f(x,y(x))\]
                \[\cfrac{dy}{dx}=g(x)H(y)\]
                \[\cfrac{dy}{dx}=\cfrac{g(x)}{h(y)}\]
                \[h(y)dy=g(x)dx\]
                \[\int h(y) dy=\int g(x) dx\]
                \begin{equation}
                    \begin{aligned}
                        \int h(y)\ dy&=\int h(y(x))\cfrac{dy}{dx} dx\\
                        &=\int h(y(x))\cfrac{g(x)}{h(y(x))} dx\\
                        &=\int g(x)\ dx
                    \end{aligned}
                \end{equation}
            \paragraph{Unlimited Population Growth}
            \paragraph{Radioactivity}
            \[\text{Half-life}=\cfrac{\ln 2}{k}\]
            \paragraph{Heat Transfer: Newton's Law of Cooling}
            \[\cfrac{dH}{dt}=-k(H-H_s)\]
            \[\cfrac{dy}{dt}=\cfrac{d}{dt}(H-H_s)=\cfrac{dH}{dt}-\cfrac{d}{dt}(H_s)\]
            \[=\cfrac{dH}{dt}\]
            \[=-k(H-H_s)\]
            \[=-ky\]
\end{document}
